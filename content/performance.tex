%%% performance.tex: -*- LaTeX -*-  DESCRIPTIVE TEXT.
%%%
%%% Copyright (c) 2017 Brian J. Fox & Orchid Labs, Inc.
%%% Author: Brian J. Fox (bfox@meshlabs.org)
%%% Author: A truckload of others
%%% Birthdate: Tue Oct 10 12:08:14 2017.

In this section we examine how the system will function as the
number of users grows.

\subsection*{Algorithmic Performance}

Broadly, there are three parts to the \Orchid{} Protocol: Ethereum-based
payments, manifolds, and \tOM{}.

Ethereum-based payments scale with Ethereum as normal transactions.
Having reviewed the Ethereum system design, we are confident that even
if the \Orchid{} Network is extremely successful, and becomes a
significant percentage of the Ethereum's total transaction volume,
this component will function within design tolerances.

Manifolds are chains of bandwidth sellers (Relays and Proxies)
all of which have performance characteristics independent of the total number of
\Orchid{} Network participants.

The core operations of \tOM{} are based on the well-studied Chord
DHT. The number of connections that Peddlers must maintain grows at a
rate of $O(log{}(n))$, to a maximum of 256 connections. Queries on the
network require $O(log{}(n))$ hops. Although these operations do become
more burdensome as the network increases in size, we do not believe
any significant impact on performance will result.

\subsection*{Allocation of Scarce Resources}

The \Orchid{} Protocol is built around tokens. These tokens will allow,
through price discovery, for graceful handling of a change in balance
between buyers and sellers.

For example, if Relays are in short supply, rather than providing all
customers with a slow experience, customers will engage in a bidding
war to determine who can use the system until the shortage is
corrected. Conversely, if Relays are in abundant supply, some Relays
may leave the system until such time as prices rise.

\subsection*{Real-World Performance}

As the software is not yet complete, we do not have concrete numbers
to provide here. On release we will update this document with the
following graphs:

\begin{enumerate}
  \item Chain setup time as a function of Orchid Market size.
  \item Orchid Market join time as a function of Orchid Market size.
  \item How quickly the price adjusts to scarcity, abundance.
  \item Add any interesting ideas here!
\end{enumerate}
