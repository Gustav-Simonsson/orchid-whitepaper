%%% firewall.tex: -*- LaTeX -*-  DESCRIPTIVE TEXT.
%%%
%%% Copyright (c) 2017 Brian J. Fox & Orchid Labs, Inc.
%%% Author: Brian J. Fox (bfox@meshlabs.org)
%%% Author: A truckload of others
%%% Birthdate: Tue Oct 10 12:07:15 2017.

The above system would be of little use if only users already
possessing free and open access to the Internet could use it. In this
section we will discuss features which ease access for those users
whose Internet access is provided by their attacker.

Please note that if an adversary is willing to completely block all
Internet access, no defense in this area is possible. All defense
analysis in this section therefore assumes that the attacker suffers
some cost for blanket blocking, and seeks to maximize this cost in the
hope that sufficiently costly attacks will not be performed.

\subsubsection*{Bootstrapping}

One of the first attacks we anticipate firewall providers to attempt
against the \Orchid{} Network is to create a list of Entry Peddlers, and to
block all access to them. This is because if customers can not access
Entry Peddlers, they could not use the network. Complicating matters, a
competent attacker must be assumed to have any list of IP addresses
available to customers.

%% To address this initially, we will provide a service which allows
%% users to perform a ``group buy'' of a small VPS instance to act as a
%% relay in exchange for tokens. Customers would then use this relay as a
%% mechanism for accessing the rest of the network, making the security
%% analysis identical to that in Section \ref{sec:collision}, with Orchid
%% Labs also acting as a secondary ISP. Since this VPS's IP address is
%% drawn from IP addresses used to host websites generally, we anticipate
%% that blanket blocking of these IP addresses would be quite costly.

To address this initially, we will provide a service which allows
users to learn fresh Relay IP addresses in exchange for proof-of-work.
To hinder blocking of the bootstrapping back and forth itself, we will
provide access to this boostrapping service via web, email, and
popular instant messaging platforms. The user will copy/paste a
challenge from their client's options screen into the most convenient
communications mechanism, then copy/paste the reply back into the
client.

\subsubsection*{DPI, Inference, and Active Probing}

More sophisticated firewalls employ methods such as Deep Packet
Inspection (DPI; analysis of the contents of packets rather than just
the headers), timing inference (the use of aggregate statistical
measures over packet size, quantity, and timing), as well as active
probing (attempting connection with the user or the server they are
connecting to in an attempt to identify the service being provided.)

We do not anticipate the use of deep packet inspection or active
probing to provide significant information. Through our use of WebRTC,
all communication is encrypted and there are no open ports unless an
active WebRTC offer has been issued. Since this matches the behavior
of all other uses of WebRTC, this behavior can not disambiguate \Orchid{}
users.

Timing inference is potentially an effective method for detecting
\Orchid{} users, as the timing and size of web requests over an encrypted
stream are unlikely to look like other kinds of WebRTC traffic
(\cite{peekaboo}). To address this, users accessing the \Orchid{} Network
in situations where inference attacks are likely are encouraged to use
``bandwidth burning'' (Section \ref{bandwidth-burning}).

\subsubsection*{Disclosure: Ethereum Traffic}

Because the current client employs an Ethereum client to track payment
statuses, and Ethereum has its own non-hardened networking signature,
detection related to this Ethereum traffic is likely to be the weak
link. Firewall operators may simply ask ``is the computer running
Ethereum \emph{and} consuming large amounts of WebRTC traffic?''

To maintain project focus, hardening of Ethereum and/or serving
Ethereum traffic over the \Orchid{} Network is not a feature of our
initial release. We plan on addressing this in future versions, see
Section \ref{securing-eth}.
