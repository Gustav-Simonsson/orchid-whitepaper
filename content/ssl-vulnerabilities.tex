%%% ssl-vulnerabilities.tex: -*- LaTeX -*-  DESCRIPTIVE TEXT.
%%% 
%%% Copyright (c) 2017 Brian J. Fox & Orchid Labs, Inc.
%%% Author: Brian J. Fox (bfox@meshlabs.org)
%%% Author: A truckload of others
%%% Birthdate: Tue Oct 10 12:05:37 2017.

SSL and TLS are complicated protocols, receiving a constant stream of
security updates as implementation flaws are discovered.
Unfortunately, users sometimes delay upgrading their software, use
untrustworthy or poorly written software, and misconfigure their
software. To protect users as possible, the \Orchid{} Protocol provides
``sanity check'' features.

\subsection{SSL Downgrade Attacks}

In so-called \emph{SSL Downgrade Attacks}, the attacker causes a
secure connection to use poor quality encryption
(\cite{ssl-downgrade}). To perform this attack, the attacker simply
removes mention of more secure encryption methods supported by the
client from the initial key negotiation packets. To prevent this
attack, the \Orchid{} Client automatically does the inverse where
possible -- it removes mention of insecure options from the key
negotiation packet (an ``SSL upgrade'' attack.)

\subsection{Old Browsers and Phone Apps}

SSL and TLS security vulnerabilities are periodically found and
patched in web browsers. However, not all users can be assumed to use
up-to-date browsers. A similar situation occurs with mobile phone
apps, where developers sometimes omit things like SSL certificate
validation.

To address these issues, the \Orchid{} Client automatically verifies
certificate chains using an up-to-date copy of ``Boring SSL'' -- the
open source SSL library used in Google Chrome.

%% TODO: we need to flesh this out once we have an SSL expert on hand.

%% \subsection{Mitigation of 8.1.1 and 8.1.2}

%% \begin{enumerate}
%% \item Check SSL Certificates
%% \item Check SSL Versions, Cipher Suites
%% \item Check Basic Constraints
%% \item “Boring SSL”
%% \end{enumerate}


