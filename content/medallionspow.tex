%%% medallions.tex: -*- LaTeX -*-  DESCRIPTIVE TEXT.
%%% 
%%% Copyright (c) 2017 Brian J. Fox & Orchid Labs, Inc.
%%% Author: Brian J. Fox (bfox@meshlabs.org)
%%% Author: A truckload of others
%%% Birthdate: Tue Oct 10 12:02:20 2017.

Building on the high-level Medallion specification in Section \ref{medallion-spec}, this appendix servers to lay out a more exact definition of a Medallion and its generation. Note that this appendix builds on notation used in \cite{Equihash}. The list below is sorted by type,
\begin{description}[align=right, labelwidth=2.7cm]
	\item[$t$ (uint)] -- unix time with precision $p$; precision must be at least on the order of 100ms
	\item[$sk_m$ (uint)]  -- a secret key $x$ chosen at random
    \item[$e_t$ (uint)]  -- the Etherium block digest (aka block hash) at time $t$ 
    \item[$h(y)$ (uint)]  -- the digest of Keccak with input $y$
    \item[$seed$ (uint)]  -- $h(e_t|sig(sk,e_t))$
    \item[$pk_m$ (tuple)]  -- a public key $x*G$ on the elliptic curve $\mathcal{C}$ with basepoint $G$ and order $N$
    \item[$sig(sk, r)$ (tuple)]  -- ECDSA signature of $r$ using secret key $sk$
    \item[$\mathcal{F}_{n,k}(x)$ (struct)] -- $\{n, k, x, i_0, ..., i_{2^{k}} \}$ : $n,k,x,i_j \in\mathbb{Z}_{|h|}$
	\item[$\mathcal{M}$ (struct)] -- $\{t, e_t, pk_m, sig(sk,e_t), \mathcal{F}_{n,k}(seed)\}$
\end{description}

There are two proposed methods for generating Medallions. The first is non interactive and requires that the generate of a Medallion have the current Etherium block digest as well as a public key pair. The Equihash proof-of-work given below is simplified.

\begin{algorithm}[H]

  \SetKwInOut{Input}{Let}
  \SetKwInOut{Output}{Return}

   \textbf{Prep Step:}\\
    ~~$sk\gets$ random$\in \mathbb{Z}_{|h|}$\\
    ~~$pk\gets sk*G$\\
    ~~$sig(sk,e_t)\gets$ ECDSA$(sk,e_t)$\\
    ~~$seed\gets h(e_t, sig(sk,e_t))$ \\
   \textbf{Equihash proof-of-work:}\\
    ~~\textit{set} difficulty $(n,k)$\\
    ~~\textit{set} counter $i_1 = seed$\\
    ~~\textit{set} $\{i_j\}$ of $2^k$ items\\
    ~~\textbf{while\{}$h(i_1) \oplus h(i_2) \oplus ...\oplus h(i_{2^k}) \neq 0$\textbf{\}\{}\\
     ~~~~\textit{build} bigger list of $\{i_j\}$\\
     ~~~~\textit{find} subsets of colliding $\{i_j\}$\\
     ~~~~\textit{sort} \{$h(i_j)$\}\\
    ~~\textbf{\}}\\
    \Output{$t, e_t, pk, sig(sk,e_t), \{i_j\}$}
    
   \caption{Non-interactive Medallion Generation}
\end{algorithm}

The second method build on the first and adds the requirement of peddlers withing \tOM{} to participate in Medallion construction. By requiring participation from \tOM{}, a challenge-response type protocol is created and may be likened to a primitive proof-of-time {\color{red}[REFERENCE]}. In this scheme, there are $m$ actors, a Medallion generator, \textit{Alice}, and a community of peddlers on \tOM{} denoted as $p_i \in$\{\textit{Bob, Chris, Dana, ...}\}. \textit{Alice} will interact with an entry peddler \textit{Bob} who will compute $sig(sk_{Bob}, e)$ and return the signature to \textit{Alice}. If further participation is required by an \Orchid{} Market, \textit{Bob} will contact $m$ random peddlers who will return $sig(sk_{p_i}, e)$ to \textit{Alice} through  \textit{Bob}. \textit{Alice} will then compute $seed$ using \{$sig(sk_{p_i}, e)$\} as additional input and return $\mathcal{M}$ to \textit{Bob}. {\color{red}There are better and more elegant ways of doing this but they might be computationally expensive. Will look into it more.}

\begin{algorithm}[H]

  \SetKwInOut{Input}{Let}
  \SetKwInOut{Output}{Return}
  
  \textbf{Prep Step:}\\
    ~~$sk\gets$ random$\in \mathbb{Z}_{|h|}$\\
    ~~$pk\gets sk*G$\\
    ~~$sig(sk,e_t)\gets$ ECDSA$(sk,e_t)$\\
   \textbf{Interactive Step:}\\
   ~~Do stuff...\\
    ~~$seed\gets h(e_t, \{sig(sk_{p_i},e_t))\}$ \\
  \textbf{Equihash proof-of-work:}\\
    ~~\textit{set} difficulty $(n,k)$\\
    ~~\textit{set} counter $i_1 = seed$\\
    ~~\textit{set} $\{i_j\}$ of $2^k$ items\\
    ~~\textbf{while\{}$h(i_1) \oplus h(i_2) \oplus ...\oplus h(i_{2^k}) \neq 0$\textbf{\}\{}\\
     ~~~~\textit{build} bigger list of $\{i_j\}$\\
     ~~~~\textit{find} subsets of colliding $\{i_j\}$\\
     ~~~~\textit{sort} \{$h(i_j)$\}\\
    ~~\textbf{\}}\\
    \Output{$t, e_t, pk, sig(sk,e_t), \{i_j\}$}
    
   \caption{Response Oriented Medallion Generation}
\end{algorithm}

% If needed to explain why time-space trade-offs are important but probably a bit much:
%
%	Time-space trade-offs are an analog to time-memory trade-offs often used in cryptanalysis. These methods take 
%	advantage of intermediate computations that allow a hard problem to be more efficiently solved if some amount 
%	of pre-computation or computed state is performed. In the case of Equihash, storing k-bit strings allows one to 
%	store possible XOR wins and perform linear algebra on the system to obtain a result. Notice that this is not
%	unlike sieving in the quadratic sieve.

\clearpage