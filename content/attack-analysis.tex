%%% attack-analysis.tex: -*- LaTeX -*-  DESCRIPTIVE TEXT.
%%%
%%% Copyright (c) 2017 Brian J. Fox & Orchid Labs, Inc.
%%% Author: Brian J. Fox (bfox@meshlabs.org)
%%% Author: A truckload of others
%%% Birthdate: Tue Oct 10 12:08:49 2017.

\subsection{Oppressive Web Applications}

Attacker Goals: Identify all \Orchid{} Relay and Proxy IP addresses.

Because \tOM{} contains all Relays and Proxies, this is the inverse
attack of the one discussed in Section \ref{sec:evasion}.

The number of forced connections on \tOM{} grows at O(log n) where
n is the network size. If an attacker holds $m\%$ of global
computation, they will learn log(n) ip addresses each time they
complete a proof-of-work computation. In $c$ epochs they will
therefore learn $1 - (1 - \frac{log(n)}{n})^c$ percent of the Relay and
Proxy IP addresses.

Readers familiar with the how the blocking of Tor traffic panned out
may worry the above describes a dire issue for the
system. Fortunately, this is not the case. Tor has around 1,000 exit
nodes, which allowed for easy filtering. In our case, largely due to
our support for whitelists, we expect to have hundreds of thousands of
exit nodes. In addition to this forming a much larger computational
challenge to unmask using the above method, blocking these IP
addresses would result in the oppressive Web Application blocking
their own users.

\subsection{Corporate Networks and ``Great'' Firewalls}

Attacker Goals: Prevent usage of the \Orchid{} network by users to whom
you provide Internet access.

Features related to this are discussed in more detail in Section
\ref{sec:evasion}. The outlook for this attacker is bleak: \Orchid{}
network usage presents as WebRTC connections relaying a fixed amount
of data. There are no open ports to probe, and no IP addresses which
can be relied on to ``out'' them.

\subsection{Passive Monitoring and Inference, perhaps with Sybil Attacks}

Attacker Goals: Learn Customer IP Identification, and Website Identification.

Analysis related to this class of attack are discussed in more detail
in Section \ref{sec:collusion}. The outlook for this attacker is
bleak: the difficulty of positioning yourself in several positions of
a Chain requires possessing (and dedicating to this attack) a large
percentage of global computation power.

\subsection{Small-Time Trolling and QoS Attacks}

Synopsis: An attacker would like to cause mayhem on as much of the network as possible.

Security minded readers looking for a good time are encouraged to
perform analysis in this domain. The task here is, given a limited
budget (perhaps on the order of \$10,000 USD), to disrupt the network
as much as possible.

\subsubsection{Attacking Chains}

Attackers may try a variety of attacks here: randomly dropping
packets, providing only very slow service, providing intermittently
slow service, or simply disconnecting. In all cases, the customer
simply replaces the node in question, leading to a minor inconvenience
spread across all customers. An additional inconvenience may be caused
to other relays in the case of dropping packets, since there may be no
way of determining if A did not forward the packet of if B is lying
about not having received it. In this case the customer replaces both
Relays.

\subsubsection{Attacking \tOM{}}

Attackers have similarly many options for this situation.

One option is to improperly implement the joining protocol. We do not
view this as an attack, as in this case our ``attacker'' is simply
paying other Econs for packet forwarding.

Another option is to join \tOM{} and refuse to provide routing table
information to users, or refuse to forward packets. This will result
in some additional routing burden on $\frac{log(n)}{n}$ of \tOM{}
queries until the Peddler in question is disconnected from the
network.

A third option is to sit on \tOM{} while offering no services. In this
case auctions performed by customers become less efficient (suffering
from the loss of one participant $\frac{log(n)}{n}$ of the time).
