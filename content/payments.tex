\subsection{Orchid Payment Requirements}
\label{sec:paymentreqs}

In most payment systems, the cost of the target item is substantially greater than the cost of the transaction. In particular, the cost of the target item is much greater than the cost associated with transferring funds from one party to another. Such is the case with most Internet purchases and networking cost may ignored as an almost trivial cost. In the \Orchid{} Network however, the cost of target item is bandwidth. Namely, each packet being sent over the wire has an associated cost. Thus if the transaction costs for sending payment is as low as the cost of a single packet, these costs would be equal. This of course, would break the economic assumptions of the \Orchid{} Protocol. 

Since we wish to sell bandwidth with arbitrary precision and \textbf{require} transaction fees to be arbitrarily low, we require a new form of payment system that enables users to pay for arbitrarily amounts of relayed traffic with minimal transaction costs. We now need a payment system with arbitrarily low transaction costs and arbitrary bandwidth divisibility. Moreover, the purpose of the \Orchid{} Protocol is to substantially reduce Internet surveillance and censorship. Thus additional requirements to our payment mechanism must include: uncensorablity, anonymity, and no reliance on trusted third parties. That is to say that even if the underlying network is resistant to surveillance and censorship but the payment mechanism is not, then the system is exploitable  and users may be censored or tracked. Similarly, relying on trusted third parties would expose the Orchid Network to interference from well motivated or powerful entities who can influence payment providers.

Thus, the requirements for Orchid Payments are:
\begin{enumerate}
\item \textit{Economic Viability}, making payments should be arbitrarily cheap.
\item \textit{Unforgeability}, only the owner a payment token should be able to use it for payments.
\item \textit{Availability}, no entity can prevent sending Orchid payments nor prevent receipt of payments.
\item \textit{Irreversibility}, it should be impossible for \textbf{any} entity to reverse past payments.
\item \textit{Anonymity}, participants should be uncorrelated to account addresses, payment amounts, or time.\footnote{Ideally, anonymity should hold not only against malicious observers, but also if the sender or recipient is malicious.}
\end{enumerate}

We discuss potential solutions for payments the following sections. We will argue that The Orchid Payments (section \ref{sec:orchid-payments}) fulfill all but the anonymity requirement.

\subsection{Traditional Payments}

In current financial payment systems, transactions are settled through negotiations between two or more entities such as banks or payment service providers\cite{PSP} using protocols such as ISO/IEC 7816\cite{ISO7816} for payment cards and EBICS\cite{EBICS} for bank payments. Such protocols run on networks such as SWIFT\cite{SWIFT} and NYCE\cite{NYCE} to support both national and international transactions. The entities forming these networks each mantain their own ledgers and continously update them from electronic payment receipts as well as manual reconcilitation\cite{Reconcil}.

Connecting to traditional payment networks typically requires special licenses in most jurisdictions as well as case-by-case business agreements between connecting entities. The resulting global financial network can be seen as an permissioned ad-hoc mesh of connecting businesses and a mix of protocols and networks. Each ledger represents a single point of failure, lacks cryptographic integrity and can be arbitrarily modified at the whims of the controlling business entity.

While classical payment protocols typically do not in of themselves define transaction fees, the entities running the protocols add fees on top. Per-transaction fees can range from a few cents for payment card transactions\cite{CardFees1} up to \$75 for international wire transfers\cite{WireTransfers1}. Many systems, instead or in addition, charge a percentage fee of the transacted amount, which can amount to as much as 13\% for bank transfers\cite{WireTransfers2} and 3.5\% for payment cards\cite{CardFees2}.

As traditional payments depend on trusted parties, they are virtually impossible to use for the Orchid Network without sacrificing the properties we require. In particular, reversibility is present by design in the form of reversal transactions\cite{CardProcessing}. Transactions are generally hard to forge, but credit card fraud is common and identity theft or hacking can lead to compromised user accounts. Moreover, these payment systems provide only partial availability, as they tend to malfunction at inconvenient times and suffer downtime on a regular basis. Anonymity is lacking as the trusted parties managing the payment typically have not only records of the sender, recipient, amount and time of payment but also often identity information about the sender. Finally, as we will see in the following sections, transaction fees on the order of traditional payments would be prohibitively expensive in the Orchid Network.

\subsection{Blockchain Payments}

Bitcoin revolutionized the status quo of traditional payment systems and continues to disrupt global markets for payments and international transfers. Bitcoin is a global network and protocol unaware of geographical boundaries. Applying public-key cryptography, transactions transfer bitcoin amounts between addresses generated by the users themselves, without the need for any trusted party. Users generate keypairs where a hash of the public key can be used as a payment address, requiring the private key to sign transfers from the address\cite{BitcoinTxs}. Bitcoin payments are unforgeable and irreversible\cite{BitcoinReview} (within a reasonable time to account for block confirmations). The Bitcoin network has seen minimal downtime since its inception and other than unlikely active censorship by miners (discussed further in section \ref{sec:oct}) it can be seen as generally available. Bitcoin payments are pseudo-anonymous and the level of anonymity depends to a large extent on how the network is used\cite{BitcoinAnon}.

In general, decentralized cryptocurrencies allow humans and computer systems alike, for the first time in history, to transact value without trusted third parties - a crucial requirement for incentivized, distributed overlay networks such as Orchid.

Transaction fees in Bitcoin are not determined by the transaction amount but rather by the size of the transaction data structure multiplied by a factor configured by the sender. Until 2017, average transaction fees remained well below \$1, but in February of 2017 fees rapidly rose as the Bitcoin network reached maximum transaction capacity. Average fees rose\cite{BitInfoChartsBTC} to as high as \$8, rendering applications relying on low fees infeasible on the Bitcoin network.

The Ethereum network, also rooted in public-key cryptography and secured by proof-of-work like Bitcoin, derives the same properties of unforgeability, availability and (non-classic) irreversibility. Ethereum has a higher and dynamically adjustable transaction capacity, and the network has seen low fees since its launch in 2015. However, due to increased number of transactions as well as price growth of Ethereum's underlying native token, Ether, transaction fees (known as \emph{gas}) have grown\cite{BitInfoChartsETH} to an average of \$0.20 with peaks up to \$1.00. Transactions executing smart contract code cost even more, in proportion to how much computation is performed.

The increase in transaction fees in popular, public blockchain networks inhibit their potential for handling micropayments directly, pushing micropayments to 2nd layer solutions such as payment channels.

\subsection{Blockchain-Based Probablistic Micropayments}

{\color{red}We will summarize our [IDEA] here but details can be found in [SECTION]} To easier convey the core idea of how probablistic payments can be applied to blockchain protocols, we will here gloss over several details. A formal description of the MICROPAY1 scheme is available in the cited original paper, and the Orchid probablistic payment scheme is formalized in section \ref{sec:orchid-payments}

Pass and Shelat describes MICROPAY1\cite{Micropayments}, where digital signatures and a commitment scheme are combined to engineer release conditions including a random outcome of an exact probability. The sender first makes a ``deposit'' by transfering bitcoin to an escrow address of a newly generated key. Then, the recipient (merchant in MICROPAY1 terms) picks a random number and sends a commitment over this number to the sender. Alongside the commitment the recipient also provide a new Bitcoin address. The sender also picks a random number and signs the concatenation of this number (in plaintext), the commitment from the recipient and other payment data such as the payment destination address provided by the recipient.

Verification of the resulting ticket involves checking that the recipient commitment matches the number they reveal, as well as verifying the signature from the sender matches the address of the bitcoin deposit. If the last two digits of XOR of the random numbers from the sender and recipient are 00 then the ticket is a win, and can be spent by the recipient.

Intuitively, we can think of the ``coin toss'' in this scheme as unbiased unless the sender can break the binding property of the commitment (or forge a signature), or if the user can break the hiding property of the commitment.

Note that the sender can ``double spend'' their deposit by issuing tickets to multiple recipients in parallel or front-run the recipient by broadcasting a spend when a ticket claim is seen from the recipient. The authors of MICROPAY1 discuss how this can be resolved by a ``penalty escrow'', a second amount deposited by the sender that can be spent back to the sender at some future time and until then  ``slashed'' or ``burned'' by anyone who can submit two valid tickets for the same payment escrow. This prevents the sender colluding with the recipient or acting as it's own recipient.

The authors of MICROPAY1 construct iterative improvements in MICROPAY2, and MICROPAY3 where a trusted party is introduced to perform some computational validation steps on the ticket and release a signature if the computations are correct.

\subsection{Orchid Payment Scheme}
\label{sec:orchid-payments}

Now that we have located a suitable abstraction for our payments, the question becomes: how should they be implemented?

Alongside the requirements discussed in section \ref{sec:paymentreqs} we also want to satisfy:
\begin{itemize}
  \item $Reusability$, the method for constructing each new ticket must not require new transaction fees or new on-chain transactions for each ticket, as otherwise transaction fees will once again be an issue.
  \item Double spending must be prevented, or failing that not profitable.
  \item The system must be sufficiently performant in terms of computational cost so as not to overwhelm the cost of a packet.
\end{itemize}

Of those requirements, the last element is perhaps the most troublesome. To the best of our knowledge, no method for constructing lottery tickets based on Ethereum tokens exists which do not require computation on the order of verifying an ECDSA signature. As detailed in this section, this follows from the requirement of the sender to cryptographically prove to the recipient not only the ticket amount and probability of winning, but also that the sender's Ethereum account has a sufficient amount of Orchid Tokens locked up for the purpose of sending tickets.

For this reason, although it was not sufficient for use alone, we are forced to employ a balance-of-trade approach similar to the one mentioned above. This in turn leads to a new requirement, namely ``the balance of trade must be kept sufficiently small so as to not cause an incentive to disconnect during trade''. As this is a mechanism design issue caused by an implementation reality, let us for now focus on implementation by assuming a solution exists, and defer further discussion until section \ref{tokens-bot}.

The Orchid payment scheme is a pseudo-anonymous, probabilistic micropayment scheme inspired by MICROPAY1 and related constructs. It mitigates front-running and parallel (including double) spending attacks without the need for a trusted party by leveraging Ethereum smart contracts and slashable penalty deposits. The pseudo-anonymity of Orchid payments is equivalent to what can be achieved in regular Ethereum transactions (although Orchid clients employ additional privacy techniques such as one-time addresses and key separation between node identities and payment addresses to achieve limited anonymity).

The trusted party introduced in MICROPAY2 and MICROPAY3 can effectively be replaced by Ethereum smart contract code. The EVM allows to implement arbitrary logic (within economic bounds on the computation) for validating micropayment tickets, and provides primitives\cite{ETHSpec} for the ECDSA\cite{ECDSA} recovery operation as well as cryptographic hash functions. An detailed description of the payment scheme is discussed in Appendix {\color{red}[SECTION]}

{\color{red}[MENTION] Orchid payment ticket reduction in value after a set time T. Discuss why this is important for balance of trade}

\subsection{The Orchid Token}
\label{sec:oct}

The \Orchid{} Network is using an Ethereum-based ERC20 token in order to satisfy the payment requirement of unforgeability, availability and irreversibility. The following sections discuss how we are able to lower transaction fees for ERC20 transfers to enable sending of arbitrarily small token amounts. Payment anonymity is discussed in section \ref{sec:anon}.

The \Orchid{} Token (OCT) is used for payments within the \Orchid{} Network. The Orchid Token is a new, Ethereum-based, ERC20-compatible, fixed-supply token. The supply is fixed at \num{1e8} tokens where each token has \num{1e18} non-divisible subunits (same divisibility as Ether).

At first glance, the \Orchid{} payment system detailed in the following sections can be configured to use Ether or any ERC20 token. In fact, using Ether would simplify the ticket contract, slightly reduce gas costs and improve usability as users would only need Ether rather than having to acquire both the \Orchid{} Token and Ether (for transaction fees).

However, Ethereum is planning future protocol upgrades to allow transaction fees to be paid by arbitrary mechanisms, including ERC20 tokens \cite{ETHAbstractions} \cite{ETHSerenity}. This will remove most of the drawbacks of using a new token; there will be no difference in gas cost and users only need to acquire a single token. It is also possible to set the gas price to zero and add an ERC20 token payment to the miner (using the EVM COINBASE\cite{ETHSpec} op code) in the contract execution \cite{ETHTokenFees}. This would require explicit support from miners as they would need to configure their mining strategy to accept zero gas price and validate that the transaction execution includes an ERC20 token transfer to the coinbase address.

However, the decision to introduce a new token instead of simply using Ether is for socioeconomical, not technical, reasons. By creating a new token and making it the only valid payment option in the Orchid Network, we engineer socioeconomic effects that we believe are significant enough to warrant the increased complexity.

Further details of The Orchid Token are discussed in Appendix {\color{red}[SECTION]}.

\subsection{Orchid Gas Costs}

We have measured a gas cost of approximately 87,000 from a solidity prototype implementation of the above scheme. This cost is for the full execution of the API for ticket claiming, when called with a winning ticket as input. The ticket claiming execution includes a sub call to the Orchid ERC20 ledger $transfer$ API. The solidity implementation of all Orchid smart contracts will be open sourced after cryptographic reviews and a minimum of external security auditing.

\subsection{Censorship Resistance}

Similar to most public blockchain networks, Ethereum transactions cannot be censored unless the validator (miners in the Ethereum network) chooses to not include them in their created blocks. As blocks are mined randomly among all miners, proportionally to hash power, it would require the vast majority of miners to actively censor Orchid payments to significantly disrupt the Orchid Network. For example, even if 90\% of the hash power chooses to not include Orchid related transactions, the Orchid Network would still function with the only caveat that transactions would take, on average, ten times longer to confirm. A more severe form of censorship would be if a large group of miners, say 51\%, chooses to censor Orchid related transactions by rejecting blocks including them \cite{BitcoinEconomics}. This is valid according to the Ethereum protocol rules and effectively creates a soft-fork. However, organizing large-scale miner collusion to create such a soft-fork comes with great risk of loss of profit; if the soft-fork fails to achieve sufficient hashing power the colluding miners would miss out on their block rewards. Aside from the profit risk, we consider this possibility extremely unlikely given the decentralized nature of Ethereum miners and the lack of legal and regulatory limitations on blockchain mining strategies.

\subsection{Balance of Trade}
\label{tokens-bot}

The realities of symmetric encryption performance prevent us from sending payment with every packet {\color{red}[<--I think this is wrong; symm crypto could keep up but pub key would fail]}, and so we need a good understanding of the risks inherent in employing a ``balance of trade'' approach. We do so here in a general setting: imagine Alice and Bob wish to transact in a fully anonymous manner. Bob is to perform some task for which he charges $x$, and Alice is to pay him once every $y$ tasks. Unfortunately, the nature of anonymity is such that without prior transactions, Alice and Bob have no mechanism to trust one another. Can they cooperate?

If there is some setup cost to Alice and Bob's relationship ($S_{Alice}, S_{Bob}$ s.t. $S_{Alice} > xy, S_{Bob} > xy$), the answer is yes: running away with the money or work ceases to be economically rational, unless (1) the total amount of work Alice was seeking was $\leq xy$ or (2) the total amount of work that Bob can perform is $\leq xy$. As we will see in our discussion of \tOM{} (Section \ref{sec:market}), setup costs exist on the \Orchid{} Network which support trade imbalances in excess of \num{1e3} packets. Because sellers in \tOM{} generally pay a higher setup cost than buyers, and because Customers asymmetrically know how much work they will require, the \Orchid{} Network has Customers pre-pay.

\subsection{Anonymity}
\label{sec:anon}

The Orchid payments discussed in prior sections are as pseudo-anonymous as regular Ethereum transactions are; all transactions are public including the amount and the sender and recipient accounts. The Orchid Client aims to improve on the default pseudo anonymity of public blockchain transactions by modern wallet techniques such as one-time addresses \cite{AddressReuse} and use of HD wallets \cite{HDWallets} to provide unlinkability of payment addresses despite using a single root key.

With the Ethereum Byzantium release, it is now possible to implement linkable ring signatures with reasonable gas costs by leveraging the new EVM primitives for Elliptic Curve operations \cite{ETHRingSigs}. Combining Ethereum smart contracts with stealth addresses such as those provided by HD wallets and linkable ring signatures enables a class of mixing technologies such as the Möbius\cite{Moebius} mixing service. Möbius provides strong anonymity guarantees that are cryptographically proven using a game-based security model for mixing services. However, unlike prior mixing technologies, it provides anonymity against malicious observers and senders but not against malicious recipients. Combining services such as Möbius with Orchid probablistic micropayments brings us closer to our final requirement on payments - anonymity.

To achieve full anonymity guarantees against any malicious actor, whether observer, sender or recipient of payments, we have to look at zero knowledge technology.

%% TODO: TumbleBit - can it be implemented in EVM?

{\color{red} [MENTION HERE BUT CONSIDER MOVING TO FUTURE WORK] zk-SNARK\cite{zkSNARKs} technology as applied in the zcash network \cite{zcash-zksnarks} can provide stronger anonymity guarantees compared to ring signatures. In zcash, transactions between shielded addresses provide full anonymity of the sender, recipient and amount.}