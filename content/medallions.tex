%%% medallions.tex: -*- LaTeX -*-  DESCRIPTIVE TEXT.
%%%
%%% Copyright (c) 2017 Brian J. Fox & Orchid Labs, Inc.
%%% Author: Brian J. Fox (bfox@meshlabs.org)
%%% Author: A truckload of others
%%% Birthdate: Tue Oct 10 12:02:20 2017.

Fully decentralized, fully anonymous digital systems suffer from
attacks in which a single malicious user pretends to be thousands of
users (Sybil Attacks).

To combat this class of attack, the \Orchid{} Protocol employs
Medallions -- data which demonstrates that a given public key was in
possession of a sizable amount of computation at a given time. As
computation is an expensive resource, the use of Medallions places
budgetary limitations on a given attacker's ability to impersonate
multiple users.

\subsection{Medallion Specification}

To produce a medallion, a peer takes a public key $K$, and the most
recent Ethereum block hash $E$, then (iteratively or in parallel)
locates a salt $S$ such that $H(K, E, S) \geq N$, where $N$ is some
difficulty scaling factor.

Because it is key specific, it cannot be used to impersonate multiple
public keys. Because it is tied to an Ethereum block hash, it cannot
be precomputed.

\subsection{Selection of Proof-Type}

Readers who are familiar with other distributed market based networks will have recognized Medallions as being similar in premise to proof-of-work systems (bitcoin, etc), and may be inclined to ask: why not use proof-of-stake, proof-of-idle, or other less energetically wasteful methods for proving “realness”?

Proof-of-stake rests on the assumption that no attacker will ever control the majority of tokens. As our attack model includes oppressive governments, this can not be counted on. Even Bitcoin’s astonishing market capitalization is far less than the GDP of a modestly sized country. Making matters more complicated, in the near future we intend to extend the system to support anonymous payments, which will make detection of such a ``hostile takeover'' much more difficult. In short: we did not use proof-of-stake because we did not want to engineer a system in which our users’ right to privacy might be sold to the highest bidder.

Proof-of-space looks much more interesting. Although we are not sure that a suitable method will be located, we are exploring the possibility of using proof-of-space for an upcoming version of the \Orchid{} Protocol. This would allow old smart phones, for example, to be installed by users in their homes as Relays and Proxies. For more on this idea, see Section \ref{future:proof-of-space}.

Proof-of-idle rests on the additional assumption that periodic, synchronized proof-of-work is sufficient to demonstrate a User’s share of the global computational power. Unfortunately, while the network is in its infancy ($\leq$ 10 million Peddlers), this leads to a situation where any company in control of a supercomputing center may, with only the sacrifice of ~1\% of their computational power, take control of the network. As we expect it to be quite a while before we have sufficient numbers of Peddlers for this attack to cease being devastating, we are not using proof-of-idle for this release.
