%%% abstract.tex: -*- LaTeX -*-  
%%% 
%%% Copyright (c) 2017 Brian J. Fox & Orchid Labs, Inc.
%%% Author: David L. Salamon (dls@meshlabs.org).
%%% Author: Brian J. Fox (bfox@meshlabs.org) Tue Oct 10 11:36:56 2017.

\noindent
As methods for censoring browsing and for discovering private browsing information have become more effective, the interest in anonymization methods has increased. Unfortunately, existing approaches to unrestricted, unsurveilled Internet access such as I2P and Tor suffer from a lack of widepread adoption.  Indeed,  only a few thousand unpaid volunteers host relays and exit nodes, allowing sophisticated attackers a tractable number of nodes to monitor or otherwise compromise. We present a market based, fully decentralized, and anonymous peer-to-peer system based on “bandwidth mining” which we believe addresses this lack of relay and exit nodes by directly incentivizing participants.

\noindent
The implementation of the system described in this paper is flexible in its use of library components and specific encryption algorithms.  Future work or availability of new encryption mechanisms may result in changes to the implementation described herein.  However, the essence of the system, its purpose and its goals will remain the same.

\begin{comment}
The Internet's current structure is the product of two forces: (1)
what is easy to implement for a small to medium sized company, and
(2) the desires of mankind.

Unfortunately, not all desires of mankind have proved agreeable. For
example, as methods for censoring browsing and discovering private
browsing information have improved, consumers have found themselves
in the unenviable position of needing to decide between their
privacy and usable Internet access. Even for those willing to suffer
the usability hit, current methods to achieve unrestricted,
unsurveilled Internet access such as I2P and Tor suffer from a
tragedy of the commons – only a few thousand unpaid volunteers host
proxies and exit nodes, allowing sophisticated attackers a tractable
number of nodes to monitor or otherwise compromise.

Similarly, not all desirable services have proved easy to
implement. Perverse economic incentives stemming from widely
available free content have resulted in a news media bereft of
subscribers, slowly descending into a minute-to-minute battle for
clicks. The high cost of distributing video has resulted in
centralization, platforms, and burdens on content creators. The
difficulty of maintaining and securing a sever have resulted in less
diversity than one might have hoped for.

To address these issues, we present a high performance approach to
networking built on a foundation of micropayments, and as an initial
application of the technology, build an uncensorable, anonymous
mechanism for accessing the global Internet. It is our hope that as
the tokens gain in acceptance, many millions of websites will
incorporate the payment models described here, and consumers will
receive a monthly budget of tokens from their ISP.
\end{comment}

\smallskip
\noindent
Contributions include:
\begin{itemize}
\item A blockchain-based stochastic payment mechanism with transaction costs on the order of a packet
\item A commodity specification for the sale of bandwidth
\item A method for distributed inductive proofs in peer-to-peer systems which make Eclipse attacks arbitrarily difficult
\item An efficient security-hardened auction mechanism suited for the sale of bandwidth in circumstances where an attacker may alter their bid as part of an attack
\item A fully distributed anonymous bandwidth market
\end{itemize}
