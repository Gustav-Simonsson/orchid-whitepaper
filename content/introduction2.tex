The \Orchid{} Protocol organizes bandwidth sellers into a rigidly
structured peer-to-peer network termed \tOM{}. Customers connect to
\tOM{} and pay multiple bandwidth sellers to form a proxy chain to a
specific webserver. Proxy chains allow customers to send and receive
data from the global Internet.

[graphic 1: regular Internet connection, VPN connection, Orchid connection]

Unlike more common methods for sending and receiving data from the
global Internet, proxy chains in \tOM{} naturally separate information
about the source of data from information about its destination, or
single bandwidth seller holds both pieces of information, or knows the
identity of bandwidth sellers who do. The rigid structure of \tOM{}
further supports this separation of information by providing strong
resistance against \emph{collusion attacks} -- the ability of a group
of bandwidth sellers to overcome this separation of knowledge.

Unlike less common methods for sending and receiving data from the
global Internet, which do compartmentalize source and destination
knowledge, \tOM{} provides \emph{fixed rate relaying} to prevent
traffic analysis, and an incentive for participation not related to
the hiding or discovery of information: payment in tokens.

Before we describe the details of the system, we will review the
problem it is to solve, and the general type of solution embodied.


\subsection{The \emph{Traffic Analysis Problem}}

\textbf{Problem Statement:} Imagine you are in a cafeteria full of
mathematicians and wish to send a message to your friend across the
room without anyone else knowing that fact.



\subsection{Fixed Rate Mix Network}

Imagine you and four other cryptographers are locked in a room
together. You would like to send a message to one cryptographer, and
would like to do so without allowing

but would like to do so in a way

To incentivize the participation of non-colluding

In the event that a

In the event that many bandwidth sellers are logging usage
information, the rigid structure of the \Ordhic{} Market allows
customers to select packet forwarding paths which with a very high
probability are not controlled by the same

Using the Relay
Network, customers can send and receive data from the global Internet
uncensorably and anonymously with a very high probability.

Adherence to this structure is enforced by
neighboring peers, and the customer during usage, providing a very
high degree of certainty that all peers actually adhere to this
structure.

The \Orchid{} Protocol organizes all participating computers into a
structured peer to peer network (the \Orchid{} Market.) Customers
connect to the \Orchid{} Market and pay network participants to
forward packets; the structure of the network has properties which
provide uncensorability and anonymity with a high probability for
customers.



\subsection{The \Orchid{} Market}

The \Orchid{} Protocol assigns bandwidth sellers a random position in
a global ring. Bandwidth sellers form network connections with each
other based on their position, and verify the correctness of the
behavior of the peers they are connected to.

\subsection{Purchases}



\subsection{Document Structure}

The document is structured as a core paper which explores the network
behavior as a whole, along with several appendixes which explore
specific topics in extreme detail. Readers are encouraged to read the
main paper to understand the network function, and explore the
appendicies on an interest.
