When purchasing bandwidth, price-sensitive customers can be taken
advantage of by attackers offering very low prices. For example,
imagine a customer who plans to purchase a length 4 chain on a
lowest-bid basis. An attacker who knows this can set their prices to
the minimum possible amount, thereby achieving greater than
$\frac{a}{n}$ chance of being picked for each node in the chain.

To address this, customers using \tOM{} select a random subset of the
bandwidth selling population, then select their providers randomly
from the set of affordable chains of the required length that can be
created from that random subset.

This creates a circumstance where attackers need to ``get lucky'' both
in their assigned location, and also in the price they select relative
to the prices set by the other randomly selected sellers. Even given
this constraint, familiar market properties such as the ability of a
seller to influence demand through price changes are maintained.

\subsection{Appendix Overview}

In this Appendix, we will look at what strategies are available to
privacy minded bandwidth buyers, and what the performance of the
different strategies are.

A simplified model of \tOM{} is introduced, suited to analysis of this
problem, and several example approaches are worked out both in theory
and for concrete examples. It is our hope that readers of this section
will come away with an understanding of the trade-offs which were made
in selecting our algorithm for bandwidth purchases, and may be
dissuaded from hard coding their client to employ a different method.

\subsection{Simplified Model for Analysis}

The full complexity of \tOM{} need not be considered to evaluate
auction strategies. To simply our analysis, we introduce here a set of
assumptions about participant goals:

\begin{enumerate}
\item \textbf{Sellers}. Sellers are in possession of $r$ ``bandwidth
  slots'' to which they would like to sell access. The sole goal of
  sellers is to maximize their earnings from this source.
\item \textbf{Attackers}. Attackers are in possession of $a \geq 2$
  sellers. Their sole goal is to have a single buyer purchase more
  than one of the slots from their sellers, which is termed a
  successful attack.
\item \textbf{Buyers}. Buyers seek to buy three bandwidth slots from
  three different sellers (thereby thwarting attackers), at the lowest
  price possible.
\end{enumerate}

Rather than concern ourselves with the details of \tOM{}, we will
assume that all buyers instead possess an up-to-date list of all
current medallion holders and their current bandwidth price. We will
also not concern ourselves with the distinction between Relays and
Proxies, or the complexities introduced by whitelists and other
capability filtering.

Now that we have a basic structure, and an understanding of
participant goals, let us flesh out a game structure:

\begin{enumerate}
\item \textbf{Setup} The strategy to be used by buyers is told to all
  sellers, and all attackers. The strategy to be used by sellers,
  which may change depending on buyer strategy, is told to attackers.
\item \textbf{Phase One}. All sellers select a price. Seller prices
  are then revealed to the attackers, who then select their prices.
\item \textbf{Phase Two}. All prices are revealed to the
  buyers. Buyers are asked in random order to select up to three
  available offers from the list.
\item \textbf{Phase Three}. Profit is distributed.
  \begin{enumerate}
  \item Sellers and attackers receive money equal to their offer price
    from any buyers who selected them.
  \item Buyers receive some profit amount specific to each buyer if
    they purchased three slots without suffering an attack, less the
    price paid for those slots.
  \item Any attackers who successfully attacked a buyer receives a
    buyer-specific bounty $U_b$.
  \end{enumerate}
\end{enumerate}

The market is then an n-player game for which we seek three
strategies: what buyers should do, what sellers should do, and what
attackers should do. We have made the buyer ``go first'' on purpose in
the above design, as buyers are the least economically normal in their
needs.

Some readers may take issue with the idea that a buyer's strategy is
shared with attackers in the above game. We do this explicitly because
a motivated attacker who is initially unsuccessful in an attack may
still acquire information on the rough prices paid by a given buyer
(for example by determining the IP address and pricing of all sellers,
then monitoring the Internet connection of that buyer to determine
which bandwidth seller was used as the first hop.) Over time this
information leakage may be used to infer the strategy being employed,
hence we feel it is best to simply assume it is possessed by the
attacker for the purposes of this analysis.

\subsubsection{Success Criteria}

A successful solution will be determined by performance on the
following criteria:

\begin{enumerate}
\item \textbf{Security}. The customer is maximally protected from
  attacks given a fixed budget and chain length.
\item \textbf{Stability}. No member of any of the three groups can
  improve their personal utility by changing strategies.
\item \textbf{Economically compatible}. Sellers can raise and lower
  prices to modulate the number of purchase orders they receive,
  allowing for familiar methods to be employed for maximizing seller
  profit.
\end{enumerate}

Stability should be given special attention by those readers familiar
with optimization but unfamiliar with game theory, as there is
considerable temptation to propose group-level ``optimal strategies''
wherein the members of a group band together to protect their
interests as a whole. Although such behavior may appear rational on
its surface, the incentive structures present in fully anonymous,
distributed markets prevent them from being stable. For example, it
might appear that sellers in a situation where the total slots on
offer exceeds demand might band together and set a minimum price (a
\emph{trust} in economic terms, superrationality\cite{metamagic} in
game theoretic terms, a class revolution in Marxist terms.)
Unfortunately, because additional profit will be available to any
sellers who lower their price below the trust price, such an
arrangement is not stable. For this reason, we will not be considering
them in this analysis. This is not to rule out their eventual
application in this domain; perhaps future advances in blockchain
technology will allow for distributed verification of adherence to
such agreements.

Another potential surprise is that we attempt to explicitly determine
how an attacker should behave so as to maximally exploit buyers, and
include the stability of such strategies as a criterion for
``success.'' We do this because there is no other way to provide
bounds on the security of a given approach.

\subsection{Selection Attacks}

Before we delve too far into discussion of buyer strategy, let us
first consider the goals of the attacker, and what constitutes an
attack. If we imagine a perfectly paranoid buyer, with infinite budget
and no information other than what is contained on the list of
sellers, it is plain they can do no better than random selection. This
gives attackers a successful attack rate of $(\frac{a}{n})^2$. As this
is the best possible result, we consider a strategy affording such
odds of attack secure.

An attack in this area is then any method whereby an attacker might
increase the chance of successful attack above $(\frac{a}{n})^2$
probability.

\subsection{Candidate Strategies}

With that background out of the way, we now proceed to evaluate buyer strategies.

\subsubsection{Lowest-Price}

If the buyer elects to go with the lowest cost provider, a component
attacker will set their prices to the lowest allowed (0 tokens). This
yields a successful attack of $\frac{a}{z}$ where $z$ is the number of
nodes charging zero.

For example, consider a market with three genuine sellers each
offering a single slot at the price points: \{2, 4, 6\}, and a buyer
with maximum total price of 12. A competent attacker will enter the
market at the prices \{0, 0\}, and so will be guaranteed the sale,
putting the chance of successful attack at $1$.

\subsubsection{Price-Weighted Random}

If the buyer elects to make the chance of purchase some monotonic
function of the difference in cost of bandwidth, this results in
moderately better performance of:

$$\frac{a f(0)}{\sum_{e \in S} f(e)}$$

Returning to the above example, and using the inverse squared increase
in cost as our function, we arrive at $\frac{288}{337} \approx 85\%$
chance of successful attack. While 85\% is much better than 100\%, it
is still unsatisfying.

\subsubsection{Random Selection From Affordable Relays}

If the buyer elects to select randomly from sellers charging less than
$\frac{1}{3}$ of their maximum price, a component attacker will set
their prices to be at or below that maximum. When in doubt, the
attacker can again select a price of 0.

Returning to our example, a component attacker will enter the market
at the prices 0 and 1, or equivalent, leading to two non-attack
combinations: \{(0, 2, 4), (1, 2, 4)\}, and two attack combinations:
\{(0, 1, 2) and (0, 1, 4)\}. This yields a chance of successful attack
of $\frac{1}{2}$.

\subsubsection{Random Selection Subject to Cost}
\label{best-auction-strat}

If the buyer elects to select randomly from triples of sellers $(S_i,
S_j, S_k)$ such that the total cost is less than or equal to the
maximum cost, the attacker can then select their prices so as to (1)
maximize the number of triples $(A_i, A_j, S_k)$ which the buyer can
afford, (2) minimize the number of triples $(A_i, S_j, S_k)$ which the
buyer can afford.

Returning to our example, a competent attacker will enter the market
at the prices 1 and 4.1, or equivalent, leading to 5 non-attack
combinations: \{(1, 2, 4), (1, 2, 6), (1, 4, 6), (2, 4, 4.1), (2, 4,
6)\} and three attack combinations: $\{(1, 2, 4.1), (1, 4, 4.1), (1,
4.1, 6)\}$ Hence the chance of successful attack is $\frac{3}{8}$.

Note the effective ``crowding out'' that is accomplished by the
attacker selecting the 4.1 price point -- only one of four
combinations which include 4.1 is \emph{not} a successful attack. It
is remarkable that even after accounting for such behavior on the part
of an attacker, random selection subject to cost is still better than
discarding the seller charging 6 from consideration.

\subsubsection{Random Cost Selection, Normalized by Peer}

A natural question to ask is: what if we bias our random sample to
prevent peers from being underrepresented? Unfortunately, the answer
is attackers will use this to their advantage.

Continuing the above example, a component attacker will select 1 and
6.1 as their prices, or equivalent. Because we the buyer can only
afford 6.1 when paired with 1, adjusting the odds results in a
$\frac{3}{7}$ chance of a successful attack.

\subsubsection{Random Cost Selection, Normalized by Pair}

Another idea, similar to the one above, is to ask: what if we bias our
random sample to normalize the probability of \emph{pairs} of sellers
being selected? By controlling which pairs are rare, the attacker
again increases the liklihood of successful attack.

Continuing the above example, a component attacker will again select 1
and 6.1 as their prices, or equivalent, this time realizing a success
rate of $\frac{24}{49}$, as pairs involving 6.1 were quite
underrepresented.

\subsection{Stability Analysis}

Now that we have outlined some candidate approaches, the question
arises: are buyers incentivized to deviate from a given strategy, and
if so what happens to the security properties of each strategy as
attackers seek to exploit a mixed population of buyers?

To avoid analysis for analysis' sake, we have omitted sections for
strategies which are suboptimal/unstable from both a pricing
perspective and from a security perspective.

\subsubsection{Lowest-Price}

Lowest-price is stable against economic incentives as the price chosen
is already the lowest possible. From a security perspective, however,
it is sub-optimal and so unstable. Security-conscious buyers will use
a different strategy, presumably Random Selection Subject to Cost.

This leads to an interesting situation wherein the attacker is forced
to decide how to allocate their resources between these two types of
buyers, to the security benefit of both groups, as discussed in the
next section.

\subsubsection{Random Selection Subject to Cost}

As the inverse of the previous discussion, this strategy is stable
from a security standpoint, but not stable to economic incentives --
those buyers who are not interested in security will employ
Lowest-Cost selection.

Some readers may be surprised by the claim that this strategy is
stable from a security standpoint. Perhaps some buyers on \tOM{} will
use their knowledge about how an attacker should go about exploiting
Random Selection Subject to Cost, to create their own improved
selection method? As previously discussed, \tOM{} is not secure
against inference of buying strategies, and more troubling there is no
way of knowing the extent to which an attacker may have inferred an
alternative chosen method. Therefore for sufficiently paranoid buyers
this method is stable, as it performs best under worst-case
assumptions.

However, in as much as some number of \tOM{} buyers ignore this
advice, or simply employ the Lowest-Cost selection method, this is
good news for security conscious buyers. Any secondary attack
optimization will only cause attackers fail to optimally exploit
Random Selection Subject to Cost.

\subsection{Economic Compatibility Analysis}

We will now turn our attention to the question of seller
strategies. Our goal here is to show the extent to which the usual
sort of economic algorithms can be employed by sellers.

We have omitted sections for strategies which are suboptimal/unstable
from both a pricing perspective and from a security perspective.

\subsubsection{Lowest-Price}

As this approach is the expected case in economics, it is fully
economically compatible.

\subsubsection{Random Selection Subject to Cost}

Although it may not initially appear that this strategy is
economically compatible, when a population of buyers which do not
share a maximum price are considered, the frequency of interest in a
seller's goods takes on the familiar shape of price sensitivity.

%% [TODO: plot. Need to generate a list of power-law'd max-prices, and
%%   plot the number of buyers for a given price. Point is interest
%%   decreases if your raise the price, increases if you lower it.]

Since sellers can modulate the number of purchase orders they receive
by raising and lowering their prices in the usual way, Random
Selection Subject to Cost is economically compatible.

\subsection{Conclusion}

We have now walked through our analysis of those auction methods
suited for buyers on \tOM{}, and thereby showed how Random Selection
Subject to Cost was selected for general use.

The reason we have selected a ``pure random'' approach stems from the
assumption an attacker will both full knowledge of a buyer's strategy,
and from the assumption that attackers will pick their prices after
legitimate sellers have chosen theirs.  In this context, the best that
a biased sample can do is nothing, while at worst it allows the
attacker to increase their chance of selection. Rather than bias our
sample, we have maximized the number of options available and selected
from that space uniformly.

Readers who are worried about this increase in cost to buyers relative
to a more traditional auction model are encouraged to consider the
premium to be ``the price of security'' -- as we have seen it is
trivial to achieve a lower price by leaving oneself wide open to
attack.
